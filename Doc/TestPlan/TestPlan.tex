\documentclass[12pt, titlepage]{article}

\usepackage{booktabs}
\usepackage{tabularx}
\usepackage{hyperref}
\hypersetup{
    colorlinks,
    citecolor=black,
    filecolor=black,
    linkcolor=red,
    urlcolor=blue
}
\usepackage[round]{natbib}

\title{SE 3XA3: Test Plan\\Staroids}

\author{Team 20, Staroids
		\\ Eoin Lynagh, lynaghe
		\\ Jason Nagy, nagyj2
		\\ Moziah San Vicente, sanvicem
}

\date{\today}

%\input{../Comments}

\begin{document}

\maketitle

\pagenumbering{roman}
\tableofcontents
\listoftables
\listoffigures

\begin{table}[bp]
\caption{\bf Revision History}
\begin{tabularx}{\textwidth}{p{3cm}p{2cm}X}
\toprule {\bf Date} & {\bf Version} & {\bf Notes}\\
\midrule
Oct 22 2018 & 1.0 & Added Purpose, Test Team, Scope, a couple Acronyms, abbreviations, and symbols\\
Date 2 & 1.1 & Notes\\
\bottomrule
\end{tabularx}
\end{table}

\newpage

\pagenumbering{arabic}

\section{General Information}

\subsection{Purpose}
This document is designed to show the detailed test plan for the Staroids game. This will include a description of the testing, validation, and verification procedures that will be implemented. All the tests in this document have been created before the final implementation has been completed and any tests have actually occured, so it will be the guide followed during the testing phase of the project.

\subsection{Scope}
The scope of the test plan is to provides a basis for testing the functionality of this re-implementation of asteroids. It has the objective of proving all the functional and non functional requirements listed in the SRS document.

\subsection{Acronyms, Abbreviations, and Symbols}

\begin{table}[hbp]
\caption{\textbf{Table of Abbreviations}} \label{Table}

\begin{tabularx}{\textwidth}{p{3cm}X}
\toprule
\textbf{Abbreviation} & \textbf{Definition} \\
\midrule
POC & Proof of Concept\\
SRS & Software Requirements Specification\\
\bottomrule
\end{tabularx}

\end{table}

\begin{table}[!htbp]
\caption{\textbf{Table of Definitions}} \label{Table}

\begin{tabularx}{\textwidth}{p{3cm}X}
\toprule
\textbf{Term} & \textbf{Definition}\\
\midrule
Functional Testing & Input-Output type of testing apporach known Input, expexted Output\\
Static Testing & Just looking at code, no actual execution\\
Dynamic Testing & Testing that requries code execution\\
Structural Testing & A whitebox type of testing Approach so cases are derived from internal structure of the software\\
Automated Testing & Testing is handeled by the testing framework (JUnit) (testing done by software)\\
Manual Testing & Manual individually written test cases. (testing done by people)\\
Stress Test & Testing the limits of a system, usually refers to amounts of data the system can handle\\
\bottomrule
\end{tabularx}

\end{table}

\subsection{Overview of Document}

\section{Plan}

\subsection{Software Description}

\subsection{Test Team}
The test team for this project consists of the following members who are each responsible for writing and executing tests for modules later to be specified:\\
- Moziah San Vicente\\
- Eoin Lynagh\\
- Jason Nagy\\

\subsection{Automated Testing Approach}

\subsection{Testing Tools}

\subsection{Testing Schedule}

See Gantt Chart at the following url ...

\section{System Test Description}

\subsection{Tests for Functional Requirements}

\subsubsection{Area of Testing1}

\paragraph{Title for Test}

\begin{enumerate}

\item{test-id1\\}

Type: Functional, Dynamic, Manual, Static etc.

Initial State:

Input:

Output:

How test will be performed:

\item{test-id2\\}

Type: Functional, Dynamic, Manual, Static etc.

Initial State:

Input:

Output:

How test will be performed:

\end{enumerate}

\subsubsection{Area of Testing2}

...

\subsection{Tests for Nonfunctional Requirements}

\subsubsection{Area of Testing1}

\paragraph{Title for Test}

\begin{enumerate}

\item{test-id1\\}

Type:

Initial State:

Input/Condition:

Output/Result:

How test will be performed:

\item{test-id2\\}

Type: Functional, Dynamic, Manual, Static etc.

Initial State:

Input:

Output:

How test will be performed:

\end{enumerate}

\subsubsection{Area of Testing2}

...

\subsection{Traceability Between Test Cases and Requirements}

\section{Tests for Proof of Concept}

\subsection{Area of Testing1}

\paragraph{Title for Test}

\begin{enumerate}

\item{test-id1\\}

Type: Functional, Dynamic, Manual, Static etc.

Initial State:

Input:

Output:

How test will be performed:

\item{test-id2\\}

Type: Functional, Dynamic, Manual, Static etc.

Initial State:

Input:

Output:

How test will be performed:

\end{enumerate}

\subsection{Area of Testing2}

...


\section{Comparison to Existing Implementation}

\section{Unit Testing Plan}

\subsection{Unit testing of internal functions}

\subsection{Unit testing of output files}

\bibliographystyle{plainnat}

\bibliography{SRS}

\newpage

\section{Appendix}

This is where you can place additional information.

\subsection{Symbolic Parameters}

The definition of the test cases will call for SYMBOLIC\_CONSTANTS.
Their values are defined in this section for easy maintenance.

\subsection{Usability Survey Questions?}

This is a section that would be appropriate for some teams.

\end{document}
