\documentclass[12pt, titlepage]{article}
%!TeX spellcheck = en-US,en-DE
\usepackage{booktabs}
\usepackage{tabularx}
\usepackage{xcolor}
\usepackage{ulem}
\usepackage{hyperref}
\hypersetup{
    colorlinks,
    citecolor=black,
    filecolor=black,
    linkcolor=red,
    urlcolor=blue
}
\usepackage[round]{natbib}

\title{SE 3XA3: Test Plan\\Staroids}

\author{Team 20, Staroids
		\\ Eoin Lynagh, lynaghe
		\\ Jason Nagy, nagyj2
		\\ Moziah San Vicente, sanvicem
}

\date{\today}

%\input{../Comments}

\begin{document}

\maketitle

\pagenumbering{roman}
\tableofcontents
\listoftables
\listoffigures

\begin{table}[h]
\caption{\bf Revision History}
\begin{tabularx}{\textwidth}{p{3cm}p{2cm}X}
\toprule {\bf Date} & {\bf Version} & {\bf Notes}\\
\midrule
Oct 22 & 1.0 & Added Purpose, Test Team, Scope, a couple Acronyms, abbreviations, and symbols\\
Oct 24 & 1.1 & Added Software Description, Overview of Document, Automated Testing Approach\\
Oct 24 & 1.15 & Added functional requirement tests\\
Oct 25 & 1.2 & Added non functional tests\\
Oct 25 & 1.3 & Added Testing tools\\
Oct 25 & 1.35 & Added Testing Schedule\\
Oct 25 & 1.5 & Added Unit Testing Plan and all subsections\\
Oct 25 & 1.55 & Added PoC tests\\
Oct 26 & 1.6 & Reviewed all sections for spelling and finished up unfinished ones\\
Oct 26 & 1.65 & Added Test Plan\\
Nov 23 & 2.0 & Removed automated testing references\\
Nov 26 & 2.1 & Spell checked up to testing tools section\\
Nov 26 & 2.2 & Changed automated testing Approach\\
Nov 26 & 2.3 & Changed testing tools section\\
\textcolor{red}{Nov 28 & 2.4 & Spell checking}\\
\bottomrule
\end{tabularx}
\end{table}

\newpage

\pagenumbering{arabic}

\section{General Information}

\subsection{Purpose}
This document is designed to show the detailed test plan for the Staroids game. This will include a description of the testing, validation, and verification procedures that will be implemented. All the tests in this document have been created before the final implementation has been completed or \sout{tests} \textcolor{red}{testing} has actually occured, so it will be the guide followed during the testing phase of the project.

\subsection{Scope}
The scope of the test plan is to provide\sout{s} a basis for testing the functionality of this re-implementation of asteroids. The objective of the Test Plan is to prove all the functional and non functional requirements listed in the SRS document.

\subsection{Acronyms, Abbreviations, and Symbols}

\begin{table}[hbp]
\caption{\textbf{Table of Abbreviations}} \label{Table}

\begin{tabularx}{\textwidth}{p{3cm}X}
\toprule
\textbf{Abbreviation} & \textbf{Definition} \\
\midrule
POC & Proof of Concept\\
SRS & Software Requirements Specification\\
\bottomrule
\end{tabularx}

\end{table}

\begin{table}[!htbp]
\caption{\textbf{Table of Definitions}} \label{Table}

\begin{tabularx}{\textwidth}{p{3cm}X}
\toprule
\textbf{Term} & \textbf{Definition}\\
\midrule
Functional Testing & Input-Output type of testing approach known input, expected output\\
Static Testing & Just looking at code, no actual execution\\
Dynamic Testing & Testing that requires code execution\\
Structural Testing & A whitebox type of testing approach, cases are derived from internal structure of the software\\
Automated Testing & Testing is handled by the testing framework \sout{(Mocha and Chai) (testing done by software)}\\
Manual Testing & Manual individually written test cases (testing done by people)\\
Stress Test & Testing the limits of a system, usually refers to amounts of data the system can handle\\
\bottomrule
\end{tabularx}

\end{table}

\subsection{Overview of Document}
This Test Plan's main goal is to inform on how Staroids will be tested for correctness in its various objectives and requirements. All tools are stated and their use case is explained. All planned test cases that are used to verify the correctness of Staroids with regards to its functional and non-functional requirements are also listed.

\section{Plan}

\subsection{Software Description}
Staroids is a recreation of the HTML-5 Asteroids project created by Doug McInnes, \textcolor{red}{which} \sout{it} itself is a recreation of the Asteroids arcade game. It allows the user to pilot a space ship through a rectangular piece of space with wrapping edges. Cohabiting with the space ship, there are several asteroids and an alien. These entities are considered hostile to the space ship and will damage it if they come into contact. The space ship and alien can defend themselves from any hostiles by firing at them with a laser bullet. This will damage anything (except the shooter) that the bullet comes into contact with. Staroids allows the user to have 3 lives and keeps a running score for the current game that is based off of how many asteroids and aliens have been destroyed.

\subsection{Test Team}
The test team for this project consists of the following members who are each responsible for writing and executing tests for modules later to be specified:\\
- Moziah San Vicente\\
- Eoin Lynagh\\
- Jason Nagy\\

\subsection{Automated Testing Approach}
\sout{Automated tests are to be used for all game situations where the situation result is expected to be the same for each one. Since the result is always true, these tests can be quickly run after every Staroids edit to ensure that no game functionalities have been broken by the edits. These tests are not done automatically at game load for several reasons. Firstly, if there is an assertion error, it would be meaningless to the user and there would be no action done on the user's part. Additionally there would be time added to the startup with no benefit to the user. Therefore, all major versions of Staroids will run through the test cases before they are pushed as a final Staroids build.}
\textcolor{red}{Automated testing will not be used to prove the correctness and validate the Staroids project. It has instead been decided to take a strictly manual approach for various reasons including but not limited to: Lack of canvas environment testing framework support. The fact that since it is a video game it is being testing for feel and continuity mostly. As well as user interaction being such an integral part of the project so this should also be represented in how it is tested.}

\subsection{Testing Tools}
\sout{The tools that will be used for testing are Mocha \& Chai and JSCover. Mocha \& Chai is the unit testing framework and assertion library that will be used to facilitate our test cases outlined in this document. This will be created once consisting of all unit tests, and then can be re-run automatically after each major revision of the software in turn automating Staroids unit testing. JSCover is the code coverage tool that will be used to check how much of the Staroids game code is being executed when we the test suite is being run. The only other things that may be considered as "testing tools" could be the internet browsers that the team are going to test the game on: Safari, Edge, Firefox and Chrome.}
\textcolor{red}{Since the team is using a manual testing approach for this project, there will be no testing tools used other than the web browsers the game will be run on. Team members and third part individuals will be responsible for most of the test execution.}

\subsection{Testing Schedule}

See \href{https://gitlab.cas.mcmaster.ca/nagyj2/Staroids/tree/master/ProjectSchedule/StaroidsGantt.pdf}{Gantt Chart}

\section{System Test Description}

\subsection{Tests for Functional Requirements}

\subsubsection{States}

\paragraph{Pre-game State Tests}

\begin{enumerate}

\item{Loading into pre-game\\}
%number tests
%create tests that are meant to fail

Type: Functional dynamic manual test

Initial State: Closed

Input: Opening file

Output: In pre-game state

How test will be performed: On game start, the game should be in the pre-game state. This can be accomplished by an assertion as part of the game set-up that states that the current game state is the pre-game.

\item{Starting game\\}

Type: Functional dynamic manual test

Initial State: Pre-game state

Input: Spacebar pressed

Output: In playing state

How test will be performed: Once the pre-game screen is loaded, the spacebar registers as pressed and Staroids should enter the playing state. Staroids is allowed to go through a few intermediate states but must end on the playing state within one second.

\item{Prohibit game start on pause\\}

Type: Functional dynamic \textcolor{red}{manual} test

Initial State: Pre-game state

Input: Pause button and spacebar pressed

Output: Paused state

How test will be performed: In the pre-game state, \textcolor{red}{the game will be paused and then the start game button will be pressed}. The paused state will prohibit progression from the pre-game state to the playing state when the spacebar is pressed. \textcolor{red}{Upon un-pausing, the game shall go back to the pre-game state.}

\end{enumerate}

\paragraph{Playing State Tests}

\begin{enumerate}

\item{Permitting pause\\}

Type: Functional dynamic \textcolor{red}{manual} test

Initial State: Playing State

Input: Pause button pressed

Output: Paused state

How test will be performed: When the spacebar is \textcolor{red}{pressed}, the game shall enter the paused state. Upon \textcolor{red}{pressing} the spacebar again, the game shall return to the playing state.

\item{Displaying player information above all\\}

Type: Structural static manual test

Initial State: None

Input: None

Output: None

How test will be performed: For the player information to be displayed above all other sprites, those sprites need to be drawn last. Inspecting the code will ensure that the player, score and lives are drawn above all other sprites.

\item{Player firing and alien destruction\\}

Type: Functional dynamic \textcolor{red}{manual} test

Initial State: Playing State

Input: Player object, alien object, and fire button pressed

Output: Bullet generation, bullet movement, alien destruction

How test will be performed: \textcolor{red}{The player is maneuvered to face a alien and upon pressing the spacebar, a bullet shall originate from the forward tip of the ship and proceed towards the alien.} The bullet speed is derived from the player ship speed plus a constant. \textcolor{red}{When hit,} the alien death \textcolor{red}{action} should trigger, \textcolor{red}{which is visibly evident from the setting of the alien's death flag.}

\item{Asteroid collision with player\\}

Type: Functional dynamic \textcolor{red}{manual} test

Initial State: Playing State

Input: Player object and asteroid object

Output: Player loss of life, player location reset, asteroid destruction, asteroid separation, Post-game state (if number of lives == 0)

How test will be performed: \textcolor{red}{The player is maneuvered to come into contact with an asteroid.} When the asteroid collides with the player, both the player and asteroid trigger their death functions, setting flags to register their deaths. On player death, the lives count is decreased which can be compared to the initial value. The asteroid child array will also increase which can once again be compared to the initial value.

\item{Asteroid separation\\}

Type: Functional dynamic \textcolor{red}{manual} test

Initial State: Playing State

Input: Player and Asteroid object

Output: Production of child asteroids

How test will be performed: \textcolor{red}{Starting with a large asteroid and a player, the player shoots a bullet at the asteroid. Upon the asteroid getting hit, the} asteroid itself should become disabled and should spawn 3 children asteroids. By \textcolor{red}{shooting the children, those child asteroids get disabled and spawn another set of children. Once this last set of children is destroyed, the large astroid object will count as dead (since the main object and all sub-objects are dead) and this will decrement the alive asteroid counter.}

\item{Withholding player spawn\\}

Type: Functional dynamic \textcolor{red}{manual} test

Initial State: Pre-game state

Input: Spacebar pressed

Output: Delayed player spawning

How test will be performed: When the player changes from the pre-game state to the playing state, the player should not spawn if an asteroid is within 100 pixels. If there is, the game shall wait for the asteroid to pass and then spawn the player

\end{enumerate}

\paragraph{Post-game State Tests}
\begin{enumerate}

\item{Post-game\\}

Type: Functional dynamic manual test

Initial State: Post-game State

Input: The restart game button "R" is pressed

Output: Pre-game State

How test will be performed: Set up the game in the post-game state and then press the "R" key and check if the new state after the key has been pressed is the pre-game state. It will also be tested that if any other keys are pressed no changes to the game state or game will be performed.
\end{enumerate}


\subsection{Tests for Nonfunctional Requirements}

\subsubsection{Usability and Style}

\paragraph{Player Test}

\begin{enumerate}

\item{Player test\\}
%All look and feel requirements
%readable text
%good performance
%strain on hand

Type: Functional dynamic manual test

Initial State: pre-game state

Input/Condition: An outside person

Output/Result: The outsider playing through Staroids a few times

How test will be performed: A friend or colleague will open Staroids with only the knowledge of what they can do in the game. After a few plays, the person is informed about all the functions of Staroids and how to perform them. The person then proceeds to play a few more rounds. After they are finished playing, they will be asked some \hyperref[interview:questions]{questions} with regards to the game's usability and playability.

\item{Browsers\\}

Type: Functional dynamic manual test

Initial State: Closed

Input/Condition: A developer and all major web browsers

Output/Result: An operational Staroids instance

How test will be performed: Staroids is to run on any major browser, so a developer will open Staroids using all major browsers and play through a few games to ensure that the game is functioning as expected. There should be no functional differences between the browsers, but performance is allowed to vary. The developer may judge whether a possible performance decrease is unacceptable.

\end{enumerate}

\paragraph{Informational Tests}

\begin{enumerate}

\item{State change\\}

Type: Functional dynamic manual test

Initial State: pre-game state

Input/Condition: A person to advance the game states

Output/Result: Insurance that any relevant game prompts are shown to the user

How test will be performed: A member of the Staroids team will start Staroids and advance through the game states manually to ensure that all states have on screen prompts to inform the user about their input options. In the pre-game state, the ship controls, pause button and start button must all be shown to the user on screen. The playing state does not need to show anything. The post-game state must show the restart button and lastly the pause state must show the un-pause button.

\end{enumerate}

\subsubsection{Internal Requirements}

\paragraph{Documentation}

\begin{enumerate}

\item{JSDoc Documentation Inspection\\}

Type: Structural static manual test

Initial State: Closed game

Input/Condition: Staroids source code and an inspector

Output/Result: Verification of proper JSDoc documentation

How test will be performed: To ensure all functions and classes are properly commented in accordance to JSDoc, a developer of Staroids will need to inspect the Staroids source code to ensure that all functions are documented and documented correctly. The developer will also ensure that all the modules contain only elements that are related to the game's operation.

\item{JSDoc Documentation Compilation\\}

Type: Structural static automatic test

Initial State: Closed game

Input/Condition: JSDoc executable and the Staroids modules

Output/Result: Properly created JSDoc documentation

How test will be performed: In addition to an inspector looking over the JSDoc comments, the executable should also compile proper documentation based on the JSDoc commenting within the source code.

\item{Complicated Function Documentation Inspection\\}

Type: Structural static manual test

Initial State: Closed game

Input/Condition: Staroids source code and an inspector

Output/Result: Verification of commented code

How test will be performed: Similar to the JSDoc commenting inspection, all complicated pieces of code must be commented to help a reader understand what a particular piece of code is doing. An external programmer can also be brought in to ensure that the code sufficiently commented to allow an outside user to understand what is occurring.

\end{enumerate}

\subsection{Traceability Between Test Cases and Requirements}
All test cases in this document have been established based on the requirements stated in the \href{https://gitlab.cas.mcmaster.ca/nagyj2/Staroids/blob/master/Doc/SRS/SRS.pdf}{SRS} document. Each individual test case was designed based on proving at least one requirement from there.

\section{Tests for Proof of Concept}

\subsection{Usability}

\paragraph{Browser test}

\begin{enumerate}

\item{Playability on all major browsers\\}

Type: Functional dynamic manual test

Initial State: None (Game closed)

Input: Open game with browser

Output: Game runs in pregame state on browser

How test will be performed: The test will be performed by opening the game with the specified browsers (Edge, Firefox, Safari, Chrome) and looking to see if the pregame screen shows up.

\item{Player test\\}

Type: Functional dynamic manual test

Initial State: pre-game state

Input/Condition: An outside person

Output/Result: The outsider playing through Staroids a few times

How test will be performed: A friend or colleague will open Staroids with only the knowledge of what they can do in the game. After a few plays, the person is informed about all the functions of Staroids and how to perform them. The person then proceeds to play a few more rounds. After they are finished playing, they will be asked some \hyperref[interview:questions]{questions} with regards to the game's usability and playability.

\end{enumerate}

\subsection{Sound Tests}

\paragraph{Sounds}

\begin{enumerate}
\item{Brake sound\\}

Type: Functional dynamic manual test

Initial State: Playing

Input: Brake button pressed

Output: Brake sound plays through speakers

How test will be performed: While the game is in the player state the down key will be pushed and it will be monitored to see if the break sound is heard.

\item{Fire sound\\}

Type: Functional dynamic manual test

Initial State: Playing

Input: Fire button pressed

Output: Fire sound through speakers

How test will be performed: While the game is in the player state the spacebar key will be pushed and it will be monitored to see if the fire sound is heard.

\item{Collision sound\\}

Type: Functional dynamic manual test

Initial State: Playing

Input: Collision between player and asteroid

Output: Collision sound will be played through speakers

How test will be performed: While the game is in the player state collision will be simulated and monitored to see if the fire sound is heard.

\item{Mute sound\\}

Type: Functional dynamic manual test

Initial State: Playing with mute on

Input: All sound actions will be simulated

Output: Animations still show on screen but no sounds company them

How test will be performed: The three sound actions above will be simulated and the volume of the computer will be monitured to make sure that no sounds are being produced.

\end{enumerate}

\section{Comparison to Existing Implementation}
Staroids is quite comparable to the original implementation of HTML5-Asteroids. Both have movement of the player ship and asteroids and asteroids explode into smaller asteroids when destroyed. The level mechanic of respawning asteroids is also working. The major differences between Staroids and the original is the absence of the alien and scoring. These are planned to be implemented into Staroids within a week of the Test Plan's creation. Staroids also has one feature that the original is lacking is the ability for the player to break and come to a stop.

\section{Unit Testing Plan}
%says should be more detailed -2
Mocha and Chai are to be used in Staroids unit testing. The testing suite will be a separate file that runs over top of all the other files and controls what occurs on the screen. All of the games modules are complete, so there is no need for stubs or drivers for testing. The coverage method will be a combination of  Mocha's metric as well as the Staroids team ensuring that each method has been tested at least once.

\subsection{Unit testing of internal functions}
To test Staroids functions the team will use test cases on all functions, having at least one test per function, perhaps more. These tests will be created until there is full coverage over all the internal functions. Testing will also run in developer mode, rather than having tests automatically run every time the game is executed. The developer mode will have to be enabled from the developer side, and does not require any user inputs (besides activating the mode).

\subsection{Unit testing of output files}
Staroids will not create any output files, and as such, no testing on output files will be necessary.

\bibliographystyle{plainnat}

\bibliography{SRS}

\newpage

\section{Appendix}

%This is where you can place additional information.

\subsection{Symbolic Parameters}

The definition of the test cases will call for SYMBOLIC\_CONSTANTS.
Their values are defined in this section for easy maintenance.
\textbf{Constants}\\
FPS = 30; The current frames per second\\
SHIP\_SIZE = 30; The ship size in pixels\\
TURN\_SPEED = 180; Player turn speed in degrees per second\\
SHIP\_THRUST = .2; Player thrust power in pixels per second squared \\
SHIP\_BRAKE = 0.98; player airbrake power (<0.9 = full stop 1 = no brake) \\
MIN\_SPEED = 0.1; minimum speed \\
MAX\_ACC = 2; maximum ship acceleration \\
MAX\_SPEED = 20; Maximum ship speed (velocity) \\
CVS\_WIDTH = 500; canvas width \\
CVS\_HEIGHT = 400; canvas height \\
BULLET\_EXTRA = 5; Extra velocity on bullet on top of ship's velocity \\
KILLABLE = true; Testing invulnerability \\
MAX\_ASTEROIDS = 2; Maximum amount of asteroids \\
TEST = false; experimental features \\


\subsection{Usability Survey Questions}
\label{interview:questions}

\begin{enumerate}

  \item Do the controls feel comfortable? Do they place the hand in an awkward position?
  \item Do the controls make sense?
  \item Do the buttons used make sense given their function? Are they relatively consistent with other programs of similar nature you have used?
  \item Is the game too difficult or easy? If so, how could this be corrected?
  \item Is the Staroids art style appealing? Would colour improve the experience?
  \item Is the text in the game readable?
  \item Did the game run smoothly? Was there any screen tearing or stuttering?
  \item Were there any symbols or graphics that you recognize?

\end{enumerate}

\end{document}
