\documentclass[12pt, titlepage]{article}

\usepackage{booktabs}
\usepackage{tabularx}
\usepackage{hyperref}
\hypersetup{
    colorlinks,
    citecolor=black,
    filecolor=black,
    linkcolor=red,
    urlcolor=blue
}
\usepackage[round]{natbib}

\title{SE 3XA3: Test Plan\\Staroids}

\author{Team 20, Staroids
		\\ Eoin Lynagh, lynaghe
		\\ Jason Nagy, nagyj2
		\\ Moziah San Vicente, sanvicem
}

\date{\today}

%\input{../Comments}

\begin{document}

\maketitle

\pagenumbering{roman}
\tableofcontents
\listoftables
\listoffigures

\begin{table}[bp]
\caption{\bf Revision History}
\begin{tabularx}{\textwidth}{p{3cm}p{2cm}X}
\toprule {\bf Date} & {\bf Version} & {\bf Notes}\\
\midrule
Oct 22 2018 & 1.0 & Added Purpose, Test Team, Scope, a couple Acronyms, abbreviations, and symbols\\
Oct 24 & 1.1 & Added Software Description, Overview of Document, Automated Testing Approach\\
\bottomrule
\end{tabularx}
\end{table}

\newpage

\pagenumbering{arabic}

\section{General Information}

\subsection{Purpose}
This document is designed to show the detailed test plan for the Staroids game. This will include a description of the testing, validation, and verification procedures that will be implemented. All the tests in this document have been created before the final implementation has been completed and any tests have actually occured, so it will be the guide followed during the testing phase of the project.

\subsection{Scope}
The scope of the test plan is to provides a basis for testing the functionality of this re-implementation of asteroids. It has the objective of proving all the functional and non functional requirements listed in the SRS document.

\subsection{Acronyms, Abbreviations, and Symbols}

\begin{table}[hbp]
\caption{\textbf{Table of Abbreviations}} \label{Table}

\begin{tabularx}{\textwidth}{p{3cm}X}
\toprule
\textbf{Abbreviation} & \textbf{Definition} \\
\midrule
POC & Proof of Concept\\
SRS & Software Requirements Specification\\
\bottomrule
\end{tabularx}

\end{table}

\begin{table}[!htbp]
\caption{\textbf{Table of Definitions}} \label{Table}

\begin{tabularx}{\textwidth}{p{3cm}X}
\toprule
\textbf{Term} & \textbf{Definition}\\
\midrule
Functional Testing & Input-Output type of testing apporach known Input, expexted Output\\
Static Testing & Just looking at code, no actual execution\\
Dynamic Testing & Testing that requries code execution\\
Structural Testing & A whitebox type of testing Approach so cases are derived from internal structure of the software\\
Automated Testing & Testing is handeled by the testing framework (JUnit) (testing done by software)\\
Manual Testing & Manual individually written test cases. (testing done by people)\\
Stress Test & Testing the limits of a system, usually refers to amounts of data the system can handle\\
\bottomrule
\end{tabularx}

\end{table}

\subsection{Overview of Document}
This Test Plan's main goal is to inform on how Staroids is tested for correctness about its various objectives and requirements. All tools are stated and their use case is explained. All planned test cases are also listed that are used to verify the correctness of Staroids with regards to its functional and non-functional requirements.

\section{Plan}

\subsection{Software Description}
Staroids is a recreation of the HTML-5 Asteroids created by Doug McInnes which itself is a recreation of the Asteroids arcade game. It allows the user to pilot a space ship through a rectangular piece of space with wrapping edges. Cohabiting with the space ship, there are several asteroids and an alien. These entities are considered hostile to the space ship and will damage it if they come into contact. The space ship and alien can defend themselves from any hostiles by firing at them with a laser bullet. This will damage anything (except the shooter) that the bullet comes into contact with. Staroids allows the user to have 3 lives and keeps a running score for the current game that is based off of how many asteroids and aliens have been destroyed.

\subsection{Test Team}
The test team for this project consists of the following members who are each responsible for writing and executing tests for modules later to be specified:\\
- Moziah San Vicente\\
- Eoin Lynagh\\
- Jason Nagy\\ %Why am I last :(

\subsection{Automated Testing Approach}
Automated tests are to be used for all game situations where the situation result is expected to be the same for every situation. Since the result is always true, these tests can be quickly run after every Staroids edit to ensure that no game functionalities have been broken by the edits. These tests are not done automatically at game for several reasons. Firstly, if there is an assertation error, it would be meaningless to the user and there would be no action done on the user's part. Additionally there would be time added to the startup with no benefit to the user. Therefore, all major versions of Staroids will run through the test cases before they are pushed as a final Staroids build.

\subsection{Testing Tools}

\subsection{Testing Schedule}

See Gantt Chart at the following url ...

\section{System Test Description}

\subsection{Tests for Functional Requirements}

\subsubsection{Area of Testing1}

\paragraph{Title for Test}

\begin{enumerate}

\item{test-id1\\}

Type: Functional, Dynamic, Manual, Static etc.

Initial State:

Input:

Output:

How test will be performed:

\item{test-id2\\}

Type: Functional, Dynamic, Manual, Static etc.

Initial State:

Input:

Output:

How test will be performed:

\end{enumerate}

\subsubsection{Area of Testing2}

...

\subsection{Tests for Nonfunctional Requirements}

\subsubsection{Area of Testing1}

\paragraph{Title for Test}

\begin{enumerate}

\item{test-id1\\}

Type:

Initial State:

Input/Condition:

Output/Result:

How test will be performed:

\item{test-id2\\}

Type: Functional, Dynamic, Manual, Static etc.

Initial State:

Input:

Output:

How test will be performed:

\end{enumerate}

\subsubsection{Area of Testing2}

...

\subsection{Traceability Between Test Cases and Requirements}

\section{Tests for Proof of Concept}

\subsection{Area of Testing1}

\paragraph{Title for Test}

\begin{enumerate}

\item{test-id1\\}

Type: Functional, Dynamic, Manual, Static etc.

Initial State:

Input:

Output:

How test will be performed:

\item{test-id2\\}

Type: Functional, Dynamic, Manual, Static etc.

Initial State:

Input:

Output:

How test will be performed:

\end{enumerate}

\subsection{Area of Testing2}

...


\section{Comparison to Existing Implementation}

\section{Unit Testing Plan}

\subsection{Unit testing of internal functions}

\subsection{Unit testing of output files}

\bibliographystyle{plainnat}

\bibliography{SRS}

\newpage

\section{Appendix}

This is where you can place additional information.

\subsection{Symbolic Parameters}

The definition of the test cases will call for SYMBOLIC\_CONSTANTS.
Their values are defined in this section for easy maintenance.
\textbf{Constants}\\
FPS = 30; The curent frames per second\\
SHIP\_SIZE = 30; The ship size in pixels\\
TURN\_SPEED = 180; Player turn speed in degrees per second\\
SHIP\_THRUST = .2; Player thrust power in pixels per second squared \\
SHIP\_BRAKE = 0.98; player airbrake power (<0.9 = full stop 1 = no brake) \\
MIN\_SPEED = 0.1; minimum speed \\
MAX\_ACC = 2; maximum ship acceleration \\
MAX\_SPEED = 20; Maximum ship speed (velocity) \\
CVS\_WIDTH = 500; canvas width \\
CVS\_HEIGHT = 400; canvas height \\
BULLET\_EXTRA = 5; Extra velocity on bullet on top of ship's velocity \\
KILLABLE = true; Testing invulnerability \\
MAX_ASTEROIDS = 2; Maximum amount of asteroids \\
TEST=false; experimental features \\


\subsection{Usability Survey Questions?}

This is a section that would be appropriate for some teams.

\end{document}
