\documentclass[12pt, titlepage]{article}

\usepackage{booktabs}
\usepackage{tabularx}
\usepackage{hyperref}
\hypersetup{
    colorlinks,
    citecolor=black,
    filecolor=black,
    linkcolor=red,
    urlcolor=blue
}
\usepackage[round]{natbib}

\title{SE 3XA3: Software Requirements Specification\\Staroids}

\author{Team 20, Staroids
		\\ Eoin Lynagh, lynaghe
		\\ Jason Nagy, nagyj2
		\\ Moziah San Vicente, sanvicem
}

\date{\today}

%\input{../Comments}

\begin{document}

\maketitle

\pagenumbering{roman}
\tableofcontents
\listoftables
\listoffigures

\begin{table}[bp]
\caption{\bf Revision History}
\begin{tabularx}{\textwidth}{p{3cm}p{2cm}X}
\toprule {\bf Date} & {\bf Version} & {\bf Notes}\\
\midrule
Sept 26 & 0.1 & Added team and project info\\
Sept 28 & 0.11 & Divided document work\\
Sept 28 & 0.12 & Added basis of functional requirements\\
Oct 2 & 0.13 & Added stakeholders\\
Oct 3 & 0.14 & Added some non functional requirements, off the shelf solutions and documentation\\
\bottomrule
\end{tabularx}
\end{table}

\newpage

%Eoin
% Purpose of Project
% Naming Conventions and Terminology
% The Scope of the Work and the Product
% New Problems
% Risks

%Jason
% Stakeholders
% Client
% Customers
% Other Stakeholders
% Context of the Work
% Off-the-Shelf Solutions
% Documentation and Training

%Moziah
% Mandated Constraints
% Relevant Facts and Assumptions
% Work Partitioning
% Open Issues
% Tasks
% Costs

%All
% Functional Requirements
% Non-Functional Requirements
% Ideas for Solutions
%?
% Individual Product Use Cases
% Migrations to the New Project
% Waiting Room

\pagenumbering{arabic}

This document describes the requirements for ....  The template for the Software
Requirements Specification (SRS) is a subset of the Volere
template~\citep{RobertsonAndRobertson2012}.  If you make further modifications
to the template, you should explicity state what modifications were made.

\section{Project Drivers}

\subsection{The Purpose of the Project}

\subsection{The Stakeholders}
For the development of Staroids, there are some key shareholders that have impact on what decisions are made and and in effect, have sway in the outcome of the project. The stakeholder's primary role is to ensure that Staroids is developed properly and that all teams involved in developemnt are satisfied with the project. The main stakeholder in Staroids are the developers as well as the client and the customer. In the case of the Staroids project, the Staroids team are the developers. The clients of the project are both the original HTML5 Asertoids developer and the Staroid team, and lastly the customer of the project is once again the developers and any online web game players.\\

\subsubsection{The Client}
Staroids is developed for the original creator of HTML5 Asteroids with the purpose of using the proper implementation and documentation techniques. The primary concern of the original creator is Staroids' faithfulness to the original as well as the adaptations of any edits that the creator wanted to make but did not get the chance to do. The Staroids team also takes the part of the client because the team very much wants to complete this project for themselves. It offers a chance to program in a language that is new and tackle a problem that the team has not attempted yet. The developer's main concern is that the project has a straightforward implementation method.\\

\subsubsection{The Customers}
The customers of the project are the developers again and web game users. The developers are the customers as this project is also being created for their sake as a challenge in JavaScript. As such, the developers concern as a customer is that the project provides an insightful and valuable learning opportunity. Web game players are also clients as they will also consume the project once it has been created. Their concerns are where and how the game will be played and how easy it is to get running on someone's machine.\\

\subsubsection{Other Stakeholders}
Some other stakeholders that could impact the project are advertisers and web game companies. Advertisers may look to spreak the availability of the Staroid project to new users, so they require the project to be unique, special or different in some way so that they can advertise to others and bring the Staroids project into the attention of new users. Web game companies may also be stakeholders because they may look to advertise or host Staroids on their web sites. In terms of advertising, they would be similar to the advertisers, but for hosting, the companies will need the project to meet a set of techinal requirements. The web game companies may need the project to be in a certain format, only use generic libraries or be written in a certain manner.\\

\subsection{Mandated Constraints}

\subsection{Naming Conventions and Terminology}

\subsection{Relevant Facts and Assumptions}

User characteristics should go under assumptions.

\section{Functional Requirements}

\subsection{The Scope of the Work and the Product}

\subsubsection{The Context of the Work}

\subsubsection{Work Partitioning}

\subsubsection{Individual Product Use Cases}

\subsection{Functional Requirements}
  \begin{itemize}
    \item Run on Google Chrome, Mozilla Firefox and Apple Safari browsers.
    \item The game shall contain pre-game, playing, post-game, and paused states.
    \item When initially ran, the pre-game screen shall show first.
    \item On the pre-game screen, if the play button is pressed, the playing screen shall show.
    \item On the press of the pause button during playing, the pause screen shall show.
    \item On the press of the pause button while paused, the playing screen shall show again.
    \item The playing screen shall always display the player character, score and lives.
    \item Every time the player character is hit by an enemy, the lives shall decrease by one.
    \item When the fire button is pressed, the player character will fire a projectile.
    \item If a projectile hits an enemy, the enemy will be removed, that enemy's death action will occur and the score will be incremented.
    \item If zero lives remain and the player character is hit, the game shall enter the post game screen.
    
  \end{itemize}

\section{Non-functional Requirements}

\subsection{Look and Feel Requirements}
  \begin{itemize}
    \item Staroids should have visually appealing graphics.
    \item Staroids should have intuitive controls.
    \item
  \end{itemize}

\subsection{Usability and Humanity Requirements}
  \begin{itemize}
    \item On the pre-game screen, game controls should be shown.
  \end{itemize}

\subsection{Performance Requirements}
  \begin{itemize}
    \item The playing state should not stutter or freeze.
    \item The project should always run above 60 frames per second.
  \end{itemize}

\subsection{Operational and Environmental Requirements}
  \begin{itemize}
    \item TODO
  \end{itemize}

\subsection{Maintainability and Support Requirements}
  \begin{itemize}
    \item Staroids should be sufficiently modularized so that edits to a specific aspect of the game are quick.
  \end{itemize}

\subsection{Security Requirements}
  \begin{itemize}
    \item Staroids should not have an accessor methods to any of the internal variables or game state objects and variables.
  \end{itemize}

\subsection{Cultural Requirements}
  \begin{itemize}
    \item Staroids should not use any symbols that may be considered offensive or rude in its target demographics
  \end{itemize}

\subsection{Legal Requirements}
  \begin{itemize}
    \item TODO
  \end{itemize}

\subsection{Health and Safety Requirements}
  \begin{itemize}
    \item TODO
  \end{itemize}

This section is not in the original Volere template, but health and safety are
issues that should be considered for every engineering project.

\section{Project Issues}

\subsection{Open Issues}

\subsection{Off-the-Shelf Solutions}
There are many implementations of asteroids in a variety of languages that could offer inspiration to Staroids. The core game that underlines Staroids and the others is the same, but implementation methods and add-ons are what separates them. Some games offer a more realistic physics simulation while others focus on the visual experience. The other implementations also are written in different languages for different platforms, so if a user cannot run Staroids with its JavaScript and HTML implementation, those users have alternatives. Imput methods also vary amongst the implementations.\\

\subsection{New Problems}

\subsection{Tasks}

\subsection{Migration to the New Product}

\subsection{Risks}

\subsection{Costs}

\subsection{User Documentation and Training}
Staroids is to be documented using the JSDoc 3 documentation solution. JSDoc 3 allows for inline commenting in JavaScript documents to be converted into a documentation website. Since JSDoc documentation is written in the JavaScript source code, the Staroids team can modify it simultaneously with any edits to any function. This allows for completely up to date documentation while making it easier on the developers. All functions, classes and states are to be documented using JSDoc as soon as they are created, and any edits made to the them must be reflected in the JSDoc comments.

\subsection{Waiting Room}

\subsection{Ideas for Solutions}

\bibliographystyle{plainnat}

\bibliography{SRS}


\newpage

\section{Appendix}

This section has been added to the Volere template.  This is where you can place
additional information.

\subsection{Symbolic Parameters}

The definition of the requirements will likely call for SYMBOLIC\_CONSTANTS.
Their values are defined in this section for easy maintenance.


\end{document}
