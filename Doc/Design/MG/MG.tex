%\acref
\documentclass[12pt, titlepage]{article}

\usepackage{fullpage}
\usepackage[round]{natbib}
\usepackage{multirow}
\usepackage{booktabs}
\usepackage{tabularx}
\usepackage{xcolor}
\usepackage{graphicx}
\usepackage{float}
\usepackage{hyperref}
\hypersetup{
    colorlinks,
    citecolor=black,
    filecolor=black,
    linkcolor=red,
    urlcolor=blue
}
\usepackage[round]{natbib}

\newcounter{acnum}
\newcommand{\actheacnum}{AC\theacnum}
\newcommand{\acref}[1]{AC\ref{#1}}

\newcounter{ucnum}
\newcommand{\uctheucnum}{UC\theucnum}
\newcommand{\uref}[1]{UC\ref{#1}}

\newcounter{mnum}
\newcommand{\mthemnum}{M\themnum}
\newcommand{\mref}[1]{M\ref{#1}}


\title{SE 3XA3: Software Requirements Specification\\Staroids}

\author{Team 20, Staroids
  \\ Moziah San Vicente, 400091284, sanvicem
  \\ Eoin Lynagh, 400067675, lynaghe
  \\ Jason Nagy, 400055130, nagyj2
}

\date{\today}

%\input{../../Comments}

\begin{document}

\maketitle

\pagenumbering{roman}
\tableofcontents
\listoftables
\listoffigures

\begin{table}[bp]
\caption{\bf Revision History}
\begin{tabularx}{\textwidth}{p{3cm}p{2cm}X}
\toprule {\bf Date} & {\bf Version} & {\bf Notes}\\
\midrule
Oct 29/18 & 0.1 & Added basic information\\
Nov 7/18 & 0.15 & Added anticipated changes\\
Nov 7/18 & 0.2 & Added some parts of module decomposition\\
Nov 7/18 & 0.25 & Added unlikely changes and module breakdown\\
Nov 7/18 & 0.3 & Added traceability matrix\\
Nov 7/18 & 0.35 & Completed traceability matrices\\
Nov 7/18 & 0.4 & Removed unnecessary sections\\
Nov 7/18 & 0.41 & Instantiation of text and formatting\\
\textcolor{red}{Nov 28/18} & 0.45 & Spelling corrections\\
\bottomrule
\end{tabularx}
\end{table}

\newpage

\pagenumbering{arabic}

\section{Introduction}

The Staroids Project is a re-development of the classic arcade game Asteroids, based on an open source implementation found on Github. It is meant to be run on all major browsers hence why it was chosen to be written in JavaScript. Throughout the re-development the main principles that have been kept in mind are modularity, and information hiding. The modularization of Staroids will be explained further throughout this document, and the principle of information hiding is touched on more in the MIS document, and is shown through the use of getters and setter to access state variables of objects, instead of directly being able to change them. The Module Guide is designed to specify the modular structure of the Staroids project and as well relate them to goals set out in the SRS document. It should allow both future designers and maintainers to easily identify the different aspects of the software. The document is organized starting with the \ref{SecChange} Changes section that lists both anticipated and unlikely changes of the software requirements. The \ref{SecMH} Module Hierarchy summarizes the module decomposition as constructed by the Staroids team, keeping in mind any anticipated changes that may come in the future. The \ref{SecConnection} Connection section specifies the connections between the software requirements of the project as laid out in the SRS documents, and the modules as listed below. The \ref{SecMD} Module Decomposition gives a detailed description of each module, for further specification please refer to the MIS. The \ref{SecTM} section includes the two traceability matrices, one to check the completeness of the design and modules against the requirements in the SRS, and the other to show the relations between anticipated changes and the corresponding modules that they would affect. The final section \ref{SecUse} describes the uses relations between the modules of the project.

%In the deliverable template for the Module Guide, the introduction section gives an overview of module decomposition and why it is important.

%In your document, do not copy this text...

%For the introduction section give a brief overview of the project and scope, followed by the purpose of the module guide document  and what it entails. You are certainly welcome to use the given text in your document, as long as you paraphrase it, and cite it.

% Staroids is decomposed into many modules, as is the commonly accepted approach to developing software. A module is a work assignment for a programmer or programming team~\citep{ParnasEtAl1984}. We advocate a decomposition based on the principle of information hiding~\citep{Parnas1972a}. This principle supports design for change, because the ``secrets'' that each module hides represent likely future changes. Staroids is developed with the knowledge that modifications are frequent and is developed to best support these modifications in future.

% Our design follows the rules layed out by \citet{ParnasEtAl1984}, as follows:
% \begin{itemize}
% \item System details that are likely to change independently should be the secrets of separate modules.
% \item Each data structure is used in only one module.
% \item Any other program that requires information stored in a module's data structures must obtain it by calling access programs belonging to that module.
% \end{itemize}

% After completing the first stage of Staroids, the Software Requirements Specification (SRS), the Module Guide (MG) were developed. The MG specifies the modular structure of the system and is intended to allow both designers and maintainers to easily identify the parts of the software.  The potential readers of this document are as follows:

% \begin{itemize}
% \item New project members: This document can be a guide for a new project member to easily understand the overall structure and quickly find the relevant modules they are searching for.
% \item Maintainers: The hierarchical structure of the module guide improves the maintainers' understanding when they need to make changes to the system. It is important for a maintainer to update the relevant sections of the document after changes have been made.
% \item Designers: Once the module guide has been written, it can be used to check for consistency, feasibility and flexibility. Designers can verify the system in various ways, such as consistency among modules, feasibility of the decomposition, and flexibility of the design.
% \end{itemize}

% The rest of the document is organized as follows. Section
% \ref{SecChange} lists the anticipated and unlikely changes of the software requirements. Section \ref{SecMH} summarizes the module decomposition that was constructed according to the likely changes. Section \ref{SecConnection} specifies the connections between the software requirements and the modules. Section \ref{SecMD} gives a detailed description of the modules. Section \ref{SecTM} includes two traceability matrices. One checks the completeness of the design against the requirements provided in the SRS. The other shows the relation between anticipated changes and the modules. Section
% \ref{SecUse} describes the use relation between modules.

\section{Anticipated and Unlikely Changes} \label{SecChange}

This section lists possible changes Staroids. There are two categories for changes based on the likeliness of the change. Anticipated changes are listed in Section \ref{SecAchange}, and unlikely changes are listed in Section \ref{SecUchange}. Anticipated changes are planned or probable in the foreseeable future of Staroids and unlikely changes are changes that are not planned for the life of Staroids.

\subsection{Anticipated Changes} \label{SecAchange}

Anticipated changes are the source of the information that is to be hidden inside the modules of Staroids. Each change will only require alteration of one module that contains the relevant hidden information. The approach adapted here is called design for change.

\begin{description}
\item[\refstepcounter{acnum} \actheacnum \label{acHardware}:] Staroids must be kept up to date with any new operating systems or updates to the supported internet browsers are released.
\item[\refstepcounter{acnum} \actheacnum \label{acUtilities}:] Enlarging of the playing screen and the scale of all on screen objects.
\item[\refstepcounter{acnum} \actheacnum \label{acSound}:] Player firing and destruction sounds.
\item[\refstepcounter{acnum} \actheacnum \label{acGameobject}:] Collision detection between all on screen sprites.
\item[\refstepcounter{acnum} \actheacnum \label{acGameobject}:] Relative speed of all projectiles (both player and alien shot)
\item[\refstepcounter{acnum} \actheacnum \label{acGameobject}:] Speed, locomotion, size and spawning of the alien
\item[\refstepcounter{acnum} \actheacnum \label{acGameobject}:] The shape and size of asteroids
\item[\refstepcounter{acnum} \actheacnum \label{acGamestate}:] Amount of lives given to the player when the game starts.
\end{description}

\subsection{Unlikely Changes} \label{SecUchange}

Certain elements of Staroids are designed in a way that avoids unnecessary complexity and as a result, would be less likely to require alteration. Other elements of Staroids are complex and are standard throughout the source code, so alteration of that element would require the editing of many module. Hence, it is not intended that these decisions will be changed.

\begin{description}
\item[\refstepcounter{ucnum} \uctheucnum \label{ucUtilities}:] Controls for the player and menu operations.
\item[\refstepcounter{ucnum} \uctheucnum \label{ucSound}:] The sound module's method of playing sounds and controls over existing sounds.
\item[\refstepcounter{ucnum} \uctheucnum \label{ucGameobject}:] The update structure of all in game objects and how they operate.
\item[\refstepcounter{ucnum} \uctheucnum \label{ucGameobject}:] The destructive property of the large and medium asteroids into 3 asteroids of one size smaller.
\item[\refstepcounter{ucnum} \uctheucnum \label{ucGamestate}:] The state structure of the core game.
\item[\refstepcounter{ucnum} \uctheucnum \label{ucGamestate}:] How the states transfer into one another.
\item[\refstepcounter{ucnum} \uctheucnum \label{ucGamestate}:] Method of text displaying to the screen.
\end{description}

\section{Module Hierarchy} \label{SecMH}

This section provides an overview of the Staroids' module design. Modules are summarized
in a hierarchy decomposed by secrets in Table \ref{TblMH}. The modules listed
below, which are leaves in the hierarchy tree, are the modules that will
actually be implemented.

\begin{description}
\item [\refstepcounter{mnum} \mthemnum \label{mHH}:] Hardware-Hiding Module
\item [\refstepcounter{mnum} \mthemnum \label{mBHa}:] Head Module
\item [\refstepcounter{mnum} \mthemnum \label{mBHu}:] Utilities Module
\item [\refstepcounter{mnum} \mthemnum \label{mBHs}:] Sound Module
\item [\refstepcounter{mnum} \mthemnum \label{mSDgo}:] GameObject Module
\item [\refstepcounter{mnum} \mthemnum \label{mSDgs}:] GameState Module
\end{description}


\begin{table}[h!]
\centering
\begin{tabular}{p{0.3\textwidth} p{0.6\textwidth}}
\toprule
\textbf{Level 1} & \textbf{Level 2}\\
\midrule

{Hardware-Hiding Module} & M1 \\
\midrule

\multirow{7}{0.3\textwidth}{Behavior-Hiding Module} & \\
& M3 \\
& M4 \\
& M5 \\
& M6 \\
\midrule

\multirow{3}{0.3\textwidth}{Software Decision Module} & \\
& M2 \\
\bottomrule

\end{tabular}
\caption{Module Hierarchy}
\label{TblMH}
\end{table}

\section{Connection Between Requirements and Design} \label{SecConnection}

The design of Staroids is intended to satisfy all requirements established in the SRS. This leads to Staroids being separated into separate modules that contain related information. The connection between requirements and modules is listed in Table \ref{TblRT}.

\section{Module Decomposition} \label{SecMD}

Modules of Staroids are decomposed according to the principle of ``information hiding'' proposed by \citet{ParnasEtAl1984}. The \emph{Secrets} field in a module decomposition is a brief statement of the design decision hidden by the module. The \emph{Services} field specifies \emph{what} the module will do without documenting \emph{how} to do it. For each module, a suggestion for the implementing software is given under the \emph{Implemented By} title. If the entry is \emph{OS}, this means that the module is provided by the operating system or by standard JavaScript libraries.

\subsection{Hardware Hiding Modules (\mref{mHH})}

%\begin{description}
%\item[Secrets:]%put what was here before
%\item[Services:] Serves as a virtual hardware used by the rest of the
%  system. This module provides the interface between the hardware and the
%  software. So, the system can use it to display outputs or to accept inputs.
%\item[Implemented By:] OS
%\end{description}

\begin{description}
\item[Secrets:]How the game handles user keyboard input, how the canvas is outlined.
\item[Services:] Loads scripts onto canvas, therefore giving the user a visual reference for the game to then decide how they would like to interact with it.
\item[Implemented By:] index.html
\end{description}

\subsection{Behaviour-Hiding Module}

%\begin{description}
%\item[Secrets:]The contents of the required behaviours.
%\item[Services:]Includes programs that provide externally visible behaviour of
%  the system as specified in the software requirements specification (SRS)
%  documents. This module serves as a communication layer between the
%  hardware-hiding module and the software decision module. The programs in this
%  module will need to change if there are changes in the SRS.
%\item[Implemented By:] --
%\end{description}

\subsubsection{Sound Module (\mref{mBHs})}
\begin{description}
\item[Secrets:] How sounds are played, how audio files are accessed, and the states of each audio object.
\item[Services:] Controls a plethora of options for each sound that the game might need including: muting, unmuting, pausing and un-pausing, stopping, and checking if the sound is paused.
\item[Implemented By:] sound.js
\end{description}

\subsubsection{Utilities Module (\mref{mBHu})}
\begin{description}
\item[Secrets:]The values of all constants for the game, and variables to handle user inputs of pressed keys.
\item[Services:] Functions to test for if a key is pressed. The game object handles score, lives, asteroids, as well as getters and setters for all game object attributes.
\item[Implemented By:] utilities.js
\end{description}

\subsubsection{GameObject Module (\mref{mSDgo})}
\begin{description}
\item[Secrets:] Contains base game object, along with all other secondary objects which inherit from it. This includes all the attributes specific to each object.
\item[Services:] Getters and setters for all game objects, draw functions to put them to the canvas, and interaction functions between objects such as collisions.
\item[Implemented By:] gameobject.js
\end{description}

\subsubsection{GameState Module (\mref{mSDgs})}
\begin{description}
\item[Secrets:] Contains state machine for the game as well as the main running loop of game.
\item[Services:] Sets game objects to desired values based on states, as well as updates the game though the main loop as time elapses and user inputs are sent in through the event listeners defined in this module. This module however does not process the inputs, only listens for them and then alerts the utilities module to process them into variables which can then be recognized.
\item[Implemented By:] gamestate.js
\end{description}


\subsection{Software Decision Module}

%\begin{description}
%\item[Secrets:] The design decision based on mathematical theorems, physical
%  facts, or programming considerations. The secrets of this module are
%  \emph{not} described in the SRS.
%\item[Services:] Includes data structure and algorithms used in the system that
%  do not provide direct interaction with the user.
%   Changes in these modules are more likely to be motivated by a desire to
%   improve performance than by externally imposed changes.
%\item[Implemented By:] --
%\end{description}

\subsubsection{Head Module (\mref{mBHu})}
\begin{description}
\item[Secrets:]All data shown on the screen will be pulled from this file.
\item[Services:] Combines all scripts together so they are able to access each others methods.
\item[Implemented By:] head.js
\end{description}


\section{Traceability Matrix} \label{SecTM}

Below shows the traceability matrices for Staroids. These track which requirement is fulfilled by which module and which modules would need to be changed for each anticipated change.

% the table should use mref, the requirements should be named, use something
% like fref
\begin{table}[H]
\centering
\begin{tabular}{p{0.2\textwidth} p{0.6\textwidth}}
\toprule
\textbf{Req.} & \textbf{Modules}\\
\midrule
%edit com
%mHH, mSDgo, mSDgs, mBHu, mBHs

F1 & \mref{mHH}, \mref{mSDgs}\\
F2 & \mref{mSDgs}\\
F3 & \mref{mSDgs}\\
F4 & \mref{mSDgs}\\
F5 & \mref{mSDgs}\\
F6 & \mref{mSDgs}\\
F7 & \mref{mSDgs}, \mref{mSDgo}, \mref{mBHu}\\
F8 & \mref{mSDgo}, \mref{mBHu}\\
F9 & \mref{mSDgo}\\
F10 & \mref{mSDgo}\\
F11 & \mref{mSDgo}, \mref{mSDgs}\\
F12 & \mref{mSDgo}\\
F13 & \mref{mSDgo}\\
F14 & \mref{mSDgo}\\
F15 & \mref{mSDgo}\\
F16 & \mref{mSDgs}\\
F17 & \mref{mSDgo}\\
F18 & \mref{mSDgo}\\
F19 & \mref{mSDgo}, \mref{mSDgo}, \mref{mBHs}\\
F20 & \mref{mSDgo}, \mref{mSDgo}, \mref{mBHs}\\
F21 & \mref{mSDgo}, \mref{mSDgo}, \mref{mBHs}\\
F22 & \mref{mSDgo}, \mref{mBHs}\\
\bottomrule
\end{tabular}
\caption{Trace Between Requirements and Modules}
\label{TblRT}
\end{table}

\begin{table}[H]
\centering
\begin{tabular}{p{0.2\textwidth} p{0.6\textwidth}}
\toprule
\textbf{AC} & \textbf{Modules}\\
\midrule
\acref{acUtilities} & \mref{mBHu}\\
\acref{acUtilities} & \mref{mBHu}\\
\acref{acSound} & \mref{mBHs}\\
\acref{acGameobject} & \mref{mSDgo}\\
\acref{acGameobject} & \mref{mSDgo}\\
\acref{acGameobject} & \mref{mSDgo}\\
\acref{acGameobject} & \mref{mSDgo}\\
\acref{acGameobject} & \mref{mSDgo}\\
\acref{acGamestate} & \mref{mSDgs}\\
\bottomrule
\end{tabular}
\caption{Trace Between Anticipated Changes and Modules}
\label{TblACT}
\end{table}

\section{Use Hierarchy Between Modules} \label{SecUse}

In this section, the uses hierarchy between all Staroids modules is provided. \citet{Parnas1978} said of two programs A and B that A {\em uses} B if correct execution of B may be necessary for A to complete the task described in its specification. That is, A {\em uses} B if there exist situations in which the correct functioning of A depends upon the availability of a correct implementation of B.  Figure \ref{FigUH} illustrates the use relation between the modules.

% It can be seen that the graph is a directed acyclic graph (DAG). Each level of the hierarchy offers a testable and usable subset of the system, and modules in the higher level of the hierarchy are essentially simpler because they use modules from the lower levels.

\begin{figure}[H]
\centering
\includegraphics[width=0.7\textwidth]{UsesHierarchy.png}
\caption{Use hierarchy among modules}
\label{FigUH}
\end{figure}

%\section*{References}

\bibliographystyle {plainnat}
\bibliography {MG}

\end{document}
