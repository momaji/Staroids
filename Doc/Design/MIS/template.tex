\documentclass[12pt]{article}

\usepackage{graphicx}
\usepackage{paralist}
\usepackage{amsfonts}
\usepackage{amsmath}
\usepackage{hhline}
\usepackage{booktabs}
\usepackage{multirow}
\usepackage{multicol}

\oddsidemargin 0mm
\evensidemargin 0mm
\textwidth 160mm
\textheight 200mm
\renewcommand\baselinestretch{1.0}

\pagestyle {plain}
\pagenumbering{arabic}

\newcounter{stepnum}

%% Comments

\usepackage{color}

\newcommand{\wss}[1]{\authornote{blue}{SS}{#1}}
\newcommand{\means}{\Rightarrow}
\newcommand{\s}{\mbox{ }}
\newcommand{\m}[1]{\mbox{#1}}

\title{Staroids, Module Interface Specification}
\author{Team 20, Staroids
  \\ Moziah San Vicente, 400091284, sanvicem
  \\ Eoin Lynagh, 400067675, lynaghe
  \\ Jason Nagy, 400055130, nagyj2
}



\begin{document}

\maketitle

\begin{table}[bp]
\caption{\bf Revision History}
\begin{tabularx}{\textwidth}{p{3cm}p{2cm}X}
\toprule {\bf Date} & {\bf Version} & {\bf Notes}\\
\midrule
Nov 06/18 & 0.1 & Added basic information to template\\
Nov 07/18 & 0.2 & Added Head module specification\\
Nov 08/18 & 0.3 & Added all module specifications\\
\bottomrule
\end{tabularx}
\end{table}

The following is a series of MISes for the modules that comprise the Staroids game

%==================================================
\newpage

\section*{Utilities Module}

\subsection*{Template Module}

Utilities

\subsection*{Uses}

N/A

\subsection*{Syntax}

\subsubsection*{Exported Types}

FPS=30\\
SHIP_SIZE=30\\
TURN_SPEED=180\\
SHIP_THRUST=0.2\\
SHIP_BREAK=0.98\\
MIN_SPEED=0.1\\
MAX_SPEED=20\\
MAX_ACC=2\\
CVS_WIDTH=780\\
CVS_HEIGHT=620\\
BULLET_EXTRA=5\\
KILLABLE=\{True,False\}\\
MAX_ASTEROIDS=2\\
TEST=\{True,False\}\\
KeyCode=\{UP,DOWN,RIGHT,LEFT,SPACE,M,P,R\}\\
EPOCH=1\\
Key=?\\
Text=?\\
Game=?\\

\subsubsection*{Exported Access Programs}

\begin{tabular}{| l | l | l | l |} %Key object
\hline
\textbf{Routine name} & \textbf{In} & \textbf{Out} & \textbf{Exceptions}\\
\hline
Key &  & Key & ~\\
\hline
isDown & KeyCode & $\mathbb{N}$ & ~\\
\hline
onKeydown & KeyCode & $\mathbb{N}$ & ~\\
\hline
onKeyup & KeyCode &  & ~\\
\hline
\end{tabular}

\subsection*{Semantics}

\subsubsection*{State Variables}

$d$: sequence of $\mathbb{N}$\\

\subsubsection*{State Invariant}

$\forall (c : \mathbb{N} | c \in d : c \gt 0)$

\subsubsection*{Assumptions}

\begin{itemize}
    \item Only known keys (as defined by KeyCode) will be put into the Key object as events to be processed.
\end{itemize}

\subsubsection*{Access Routine Semantics}

Key():
\begin{itemize}
    \item transition: $d :=$ seq of KeyCode
    \item output: $out := \mathit{self}$
    \item exception: None
\end{itemize}

\noindent isDown(e):
\begin{itemize}
    \item output: $e \in d \Rightarrow true \land e \notin d \Rightarrow false$
    \item exception: None
\end{itemize}

\noindent onKeydown(e):
\begin{itemize}
    \item transition: $d[e] = EPOCH$
    \item exception: None
\end{itemize}

\noindent onKeyup(e):
\begin{itemize}
    \item output: $out := d[e]$
    \item exception: None
\end{itemize}


\newpage %Text Object

\begin{tabular}{| l | l | l | l |}
\hline
\textbf{Routine name} & \textbf{In} & \textbf{Out} & \textbf{Exceptions}\\
\hline
Text & Screen, Font & Text & ~\\
\hline
norm & $String, \mathbb{Z}, \mathbb{Z}$ &  & ~\\
\hline
emph & $String, \mathbb{Z}, \mathbb{Z}$ &  & ~\\
\hline
\end{tabular}

\subsection*{Semantics}

\subsubsection*{State Variables}

$cvs$: Screen %ASK
$fnt$: Font %ASK

\subsubsection*{State Invariant}

None

\subsubsection*{Assumptions}

\begin{itemize}
  \item Before the Text object is used, the initialization function must be run first.
\end{itemize}

\subsubsection*{Access Routine Semantics}

norm($Str,x,y$):
\begin{itemize}
    \item transition: $cvs[x][y]:=screenShow(Str,\mbox{NORMAL})$
    \item exception: None
\end{itemize}

emph($Str,x,y$):
\begin{itemize}
    \item transition: $cvs[x][y]:=screenShow(Str,\mbox{EMPHASIS})$
    \item exception: None
\end{itemize}

\newpage %Game

\begin{tabular}{| l | l | l | l |}
\hline
\textbf{Routine name} & \textbf{In} & \textbf{Out} & \textbf{Exceptions}\\
\hline
Game &  & ~\\
\hline
reduceCounter & $String, \mathbb{Z}, \mathbb{Z}$ &  & ~\\
\hline
resetMute &  &  & ~\\
\hline
resetPause &  &  & ~\\
\hline
drawLives &  &  & ~\\
\hline
addScore & $\mathbb{Z}$ &  & ~\\
\hline
addSprites & OBJECT &  & ~\\
\hline
subLives & $\mathbb{Z}$ &  & ~\\
\hline
subSprites & OBJECT &  & ~\\
\hline
\end{tabular}

\subsection*{Semantics}

\subsubsection*{State Variables}

$score$: $\mathbb{N}$
$lives$: $\mathbb{N}$
$sprites$: sequence of OBJECT
$muteSound$: $\mathbb{N}$
$pauseGame$: $\mathbb{N}$

\subsubsection*{State Invariant}

None

\subsubsection*{Assumptions}

\begin{itemize}
\end{itemize}

None

\subsubsection*{Access Routine Semantics}

Game():
\begin{itemize}
    \item transition: $score = 0 \land lives = 3 \land sprites = seq. of \mbox{OBJECT} $
    \item exception: None
\end{itemize}

%==================================================
\newpage

\section*{Sound Module}

\subsection*{Uses}

AUDIO for Sound\\

\subsection*{Syntax}

\subsubsection*{Exported Access Programs}

\begin{tabular}{| l | l | l | l |}
    \hline
    \textbf{Routine name} & \textbf{In} & \textbf{Out} & \textbf{Exceptions}\\
    \hline
    Sound & ~ & Sound & ~\\
    \hline
    play & Sound & ~ & ~\\
    \hline
    isPlay & Sound & Boolean & ~\\
    \hline
    pause & Sound & ~ & ~\\
    \hline
    unpause & Sound & ~ & ~\\
    \hline
    stop & Sound & ~ & ~\\
    \hline
    mute & ~ & ~ & ~\\
    \hline
    unmute & ~ & ~ & ~\\
    \hline
    toggle & ~ & ~ & ~\\
    \hline
\end{tabular}

\subsection* {Semantics}

\subsubsection* {State Variables}

%maybe the sounds
Sound: Audio object from file

\subsubsection* {State Invariant}

%ask about state invariant of sounnd

\subsubsection* {Assumptions}

\begin{itemize}
    \item The constructor is called before other accesses
    \item The sound files are in the correct directory for the projectiles
    \item the sound files have the same name as expected.
\end{itemize}

\subsubsection* {Access Routine Semantics}

Sound():
\begin{itemize}
  \item transition: $muted := true$
  %ask about the Audio objects
\end{itemize}

play():
\begin{itemize}
  \item input: $in := x \in \mbox{Sound}$
  \item transition: $!in.muted: in.play()$
\end{itemize}

isPlay():
\begin{itemize}
  \item input: $in := x \in \mbox{Sound}$
  \item output: $out := !x.paused()$
\end{itemize}

pause():
\begin{itemize}
  \item input: $in := x \in \mbox{Sound}$
  \item transition: $in.paused := true$
\end{itemize}

unpause():
\begin{itemize}
  \item input: $in := x \in \mbox{Sound}$
  \item transition: $in.paused := !true$
\end{itemize}

stop():
\begin{itemize}
  \item input: $in := x \in \mbox{Sound}$
  \item transition: $in.paused := true && this.currentTime := 0$
\end{itemize}

mute():
\begin{itemize}
  \item input: $in := x \in \mbox{Sound}$
  \item transition: $in.muted := true$
\end{itemize}

unmute():
\begin{itemize}
  \item input: $in := x \in \mbox{Sound}$
  \item transition: $in.muted := !true$
\end{itemize}

toggle():
\begin{itemize}
  \item input: $in := x \in \mbox{Sound}$
  \item transition: $in.muted := !in.muted$
\end{itemize}

%===================================================
\newpage

\section*{GameObject Module}

\subsection*{Template Module}

GameObject

\subsection*{Uses}

N/A\\

\subsection*{Syntax}

\subsubsection*{Exported Types}

GameObject, Player, Alien

\subsubsection*{Exported Constants}

None

\subsubsection*{Exported Access Programs}

\begin{tabular}{| l | l | l | l |}
    \hline
    \textbf{Routine name} & \textbf{In} & \textbf{Out} & \textbf{Exceptions}\\
    \hline
    StackT & ~ & ~ & ~\\
    \hline
    addCard & CardT & ~ & ~\\
    \hline
    remCard & ~ & CardT & is\_empty\\
    \hline
    peek & ~ & CardT & is\_empty\\
    \hline
    size & ~ & $\mathbb{N}$ & ~\\
    \hline
\end{tabular}

\subsection* {Semantics}

\subsubsection* {State Variables}

$c$: seq of CardT\\

\subsubsection* {State Invariant}

None

\subsubsection* {Assumptions}

\begin{itemize}
    \item The constructor StackT is called for each object instance
    before any other access routine is called for that object. The
    constuctor cannot be called on an existing object.
    \item StackT can be considered empty when it is of length 0 or the
    StackT is of length 1 and peek() returns a CardT with getSuit()=\m{NAS}
    and getRank()=\m{NAR}.
\end{itemize}

\subsubsection* {Access Routine Semantics}

StackT():
\begin{itemize}
    \item transition: $c := \{\}$
    \item output: $out := \mathit{self}$
    \item exception: None
\end{itemize}

\noindent addCard(C):
\begin{itemize}
    \item transition: $c := c||C$
    \item exception: None
\end{itemize}

\noindent remCard():
\begin{itemize}
    \item transition: $c := c[1:|c|-1]$
    \item exception: $(|c|=0 \means \m{is\_empty})$
\end{itemize}

\noindent peek():
\begin{itemize}
    \item output: $out := c[0]$
    \item exception: $(|c|=0 \means \m{is\_empty})$
\end{itemize}

\noindent size():
\begin{itemize}
    \item output: $out := |c|$
    \item exception: None
\end{itemize}

%===================================================
\newpage

\section*{Board ADT Module}

\subsection*{Template Module}

Board

\subsection*{Uses}

CardADT for CardT, SuitT, RankT\\
DeckADT for DeckT\\
StackADT for StackT\\

\subsection*{Syntax}

\subsubsection*{Exported Types}

BoardT=?

\subsubsection*{Exported Constants}

None

\subsubsection*{Exported Access Programs}

\begin{tabular}{| l | l | l | l |}
    \hline
    \textbf{Routine name} & \textbf{In} & \textbf{Out} & \textbf{Exceptions}\\
    \hline
    BoardT & ~ & BoardT & ~\\
    \hline
    hasWon & ~ & $\mathbb{B}$ & ~\\
    \hline
    getStack & $\mathbb{N}$ & StackT & invalid\_index\\
    \hline
    getFree & $\mathbb{N}$ & CardT & invalid\_index\\
    \hline
    getWin & $\mathbb{N}$ & CardT & invalid\_index\\
    \hline
    setStack & $\mathbb{N}$, StackT & ~ & invalid\_index\\
    \hline
    setFree & $\mathbb{N}$, CardT & ~ & invalid\_index\\
    \hline
    setWin & $\mathbb{N}$, CardT & ~ & invalid\_index\\
    \hline
    moveColToCol & $\mathbb{N},\mathbb{N}$ & ~ & invalid\_index, stack\_empty,\\
    & & & not\_alternating\_colour, not\_decending\_rank\\
    \hline
    moveColToFree & $\mathbb{N},\mathbb{N}$ & ~ & invalid\_index, stack\_empty, occupied\_cell\\
    \hline
    moveFreeToCol & $\mathbb{N},\mathbb{N}$ & ~ & invalid\_index, is\_empty, unoccupied\_cell, \\
    & & & not\_alternating\_colour, not\_decending\_rank\\
    \hline
    moveColToWin & $\mathbb{N},\mathbb{N}$ & ~ & invalid\_index, is\_empty, not\_same\_suit, \\
    & & &not\_ascending\_rank\\
    \hline
    moveFreeToWin & $\mathbb{N},\mathbb{N}$ & ~ & invalid\_index, unoccupied\_cell, not\_same\_suit,\\
    & & & not\_ascending\_rank\\
    \hline
    isValidMoves & ~ & $\mathbb{B}$ & ~\\
    \hline
\end{tabular}

\subsection* {Semantics}

\subsubsection* {State Variables}

$col$: sequence of StackT\\
$fre$: sequence of CardT\\
$fou$: sequence of CardT\\
$dek$: DeckT\\

\subsubsection* {State Invariant}

\begin{itemize}
  \item All StackTs within $col$ must have a CardT with getSuit()=\m{NAS}
  and getRank()=\m{NAR} at the bottom (first added on).
\end{itemize}

\subsubsection* {Assumptions}

\begin{itemize}
    \item The constructor BoardT is called for each object instance
    before any other access routine is called for that object. The
    constuctor cannot be called on an existing object.
    \item Unallocated $fre$ locations are to be filled with a CardT with
    getSuit()=\m{NAS} and getRank()=\m{NAR}.
\end{itemize}

\subsubsection* {Access Routine Semantics}

BoardT():
\begin{itemize}
    \item transition: $col := \forall(c : \mbox{CardT} | c \in dek : col||c)$

    \hspace{1.8cm} $fre :=$ seq of CardT

    \hspace{1.8cm} $fou :=$ seq of CardT

    \hspace{1.8cm} $dek :=$ DeckT()
    \item output: $out := \mathit{self}$
    \item exception: None
\end{itemize}

\noindent hasWon():
\begin{itemize}
  \item output: $out := \m{BoardEmpty}(col) \land \forall(c:\m{CardT}|c \in fre : \m{FreeCellEmpty}(c)) \land forall(C:\m{CardT}|C \in fou : \m{FoundationComplete}(C))$
  \item exception: None
\end{itemize}

\noindent getStack(i):
\begin{itemize}
  \item output: $out := col[i]$
  \item exception: $(\lnot (0<=i<8) \means \m{invalid\_index})$
\end{itemize}

\noindent getFree(i):
\begin{itemize}
  \item output: $out := fre[i]$
  \item exception: $(\lnot (0<=i<4) \means \m{invalid\_index})$
\end{itemize}

\noindent getWin(i):
\begin{itemize}
  \item output: $out := fou[i]$
  \item exception: $(\lnot (0<=i<4) \means \m{invalid\_index})$
\end{itemize}

\noindent setStack(i,S):
\begin{itemize}
  \item transition: $col[i]=S$
  \item exception: $(\lnot (0<=i<8) \means \m{invalid\_index})$
\end{itemize}

\noindent getFree(i,C):
\begin{itemize}
  \item transition: $fre[i]=C$
  \item exception: $(\lnot (0<=i<4) \means \m{invalid\_index})$
\end{itemize}

\noindent getWin(i,C):
\begin{itemize}
  \item transition: $fou[i]=C$
  \item exception: $(\lnot (0<=i<4) \means \m{invalid\_index})$
\end{itemize}

\noindent moveColToCol(a,b):
\begin{itemize}
    \item transition: $col[a], col[b] := col[a].\m{remCard}(), col[b].\m{addCard}(col[a].\m{peek}())$
    \item exception: $((\lnot \m{ValidIndex}(8,8,a,b) \means \m{invalid\_index}) \lor (\m{StackEmpty}(col[a]) \means \m{stack\_empty}) \lor (\lnot \m{AlternatingColour}(col[a].\m{peek}(),col[b].\m{peek}()) \means \m{not\_alternating\_colour}) \lor (\lnot \m{DecreasingRank}(col[a].\m{peek}(),col[b].\m{peek}()) \means \m{not\_decreasing\_rank}))$
\end{itemize}

\noindent moveColToFree(a,b):
\begin{itemize}
    \item transition: $col[a], fre[b] := col[a].remCard(), fre[b]=col[a].peek()$
    \item exception: $((\lnot \m{ValidIndex}(8,4,a,b) \means \m{invalid\_index}) \lor (\m{StackEmpty}(col[a]) \means \m{stack\_empty}) \lor (\lnot \m{CellFree}(b) \means \m{occupied\_cell}))$
\end{itemize}

\noindent moveFreeToCol(a,b):
\begin{itemize}
    \item transition: $fre[a], col[b] := fre[a]=\mbox{CardT}(\mbox{NAS,NAR}), col[a].\m{addCard}(fre[a])$
    \item exception: $((\lnot \m{ValidIndex}(4,8,a,b) \means \m{invalid\_index}) \lor (\m{StackEmpty}(col[b]) \means \m{stack\_empty}) \lor (\m{CellFree}(a) \means \m{occupied\_cell})) \lor (\lnot \m{AlternatingColour}(fre[a],col[b].\m{peek}()) \means \m{not\_alternating\_colour}) \lor (\lnot \m{DecreasingRank}(fre[a],col[b].\m{peek}()) \means \m{not\_decreasing\_rank}))$
\end{itemize}

\noindent moveColToWin(a,b):
\begin{itemize}
    \item transition: $col[a], fou[b] := col[a].\m{remCard}(), fou[b]=col[a].\m{peek}()$
    \item exception: $((\lnot \m{ValidIndex}(8,4,a,b) \means \m{invalid\_index}) \lor (\m{StackEmpty}(col[a]) \means \m{stack\_empty}) \lor (\lnot \m{SameSuit}(col[a].\m{peek}(),fou[b]) \means \m{not\_same\_suit}) \lor (\lnot \m{IncreasingRank}(fou[b],col[a].\m{peek}()) \means \m{not\_ascending\_rank})$
\end{itemize}

\noindent moveFreeToWin(a,b):
\begin{itemize}
    \item transition: $fre[a], fou[b] := fre[a]=\mbox{CardT}(\mbox{NAS,NAR}), fou[b]=col[a].\m{peek}()$
    \item exception: $((\lnot \m{ValidIndex}(4,4,a,b) \means \m{invalid\_index}) \lor (\m{CellFree}(a) \means \m{occupied\_cell})) \lor (\lnot \m{SameSuit}(fre[a],fou[b]) \means \m{not\_same\_suit}) \lor (\lnot \m{IncreasingRank}(fou[b],fre[a] \means \m{not\_ascending\_rank})$
\end{itemize}

\noindent isValidMoves():
\begin{itemize}
    \item output $out := \exists(s:\m{StackT}|s \in col: \exists(c:\m{CardT}|c \in fou: \m{isIncreasingRank}(c,s.\m{peek}()) \land \m{SameSuit}(c,s.\m{peek}()))) \lor \exists(c_1:\m{CardT}|c_1 \in fre: \exists(c_2:\m{CardT}|c_2 \in fou: \m{isIncreasingRank}(c_2,c_1) \land \m{SameSuit}(c_1,c_2)))$\\
    $\lor \exists(s_1:\m{StackT}|s_1 \in col: \exists(s_2:\m{StackT}|s_2 \in col: s_1 \neq s_2 \land (isIncreasingRank(s_1.peek(),s_2.peek()) \lor isDecreasingRank(s_1.peek(),s_2.peek())) \land iAlternatingRank(S_1.peek(),s_2.peek()) \land \lnot isStackEmpty(s_1) \land \lnot isStackEmpty(s_2) ))$\\
    $\lor \exists(c_1:\m{CardT}|c_1 \in fre: \exists (s_1:\m{StackT}|s_1 \in col: (\m{AlternatingColour}(c_1,s_1.\m{peek}) \land (\m{IncreasingRank}(c_1,s_1.\m{peek}) \lor \m{DecreasingRank}(c_1,s_1.\m{peek})) \land c_1.\m{isValid}()) \lor (\lnot c_1.\m{isValid})\land \lnot \m{isStackEmpty}(s_1) )) $\\
    \item exception: None
\end{itemize}

\subsection*{Local Functions}

\noindent screenShow: $String\times\{\mbox{NORMAL},\mbox{EMPHASIS}\}$
\noindent output: $out := $ %ASK


%QUESTIONS GO HERE

% How to deal with: Drawing to screen, printing
% Can a 'refer to JSDoc Documentation' be used for simple getters and setters




\noindent ValidIndex: $\mathbb{N}_1\times\mathbb{N}_2\times\mathbb{N}_3\times\mathbb{N}_4\rightarrow\mathbb{B}$\\
\noindent output: $out := (0<=\mathbb{N}_3<\mathbb{N}_1) \land (0<=\mathbb{N}_4<\mathbb{N}_2)$\\

\noindent AlternatingColour: $\mbox{CardT}_1 \times \mbox{CardT}_2 \rightarrow \mathbb{B}$\\
\noindent output: $out := (\mbox{CardT}_1.\m{getColour}()=\m{RED} \land \mbox{CardT}_2.\m{getColour}()=\m{BLACK}) \lor (\mbox{CardT}_1.\m{getColour}()=\m{BLACK} \land \mbox{CardT}_2.\m{getColour}()=\m{RED})$

\noindent IncreasingRank: $\mbox{CardT}_1 \times \mbox{CardT}_2 \rightarrow \mathbb{B}$\\ %lower, higher
\noindent output: $out := \mbox{CardT}_1.\m{getRank}()=\mbox{CardT}_2.\m{getRank}()-1$\\

\noindent DecreasingRank: $\mbox{CardT} \times \mbox{CardT} \rightarrow \mathbb{B}$\\ %higher, lower
\noindent output: $out := \mbox{CardT}_1.\m{getRank}()=\mbox{CardT}_2.\m{getRank}()+1$\\

\noindent StackEmpty: $\mbox{StackT} \rightarrow \mathbb{B}$\\
\noindent output: $out := \m{StackT}.\m{size}()=0 \lor (\m{StackT}.\m{size}()=1 \land \lnot \m{StackT.peek.isValid}())$\\

\noindent CellFree: $\mathbb{N} \times \mbox{seq of CardT} \rightarrow \mathbb{B}$\\
\noindent output: $out := \lnot (\mbox{seq of CardT})[\mathbb{N}].\m{isValid}$\\

\noindent SameSuit: $\mbox{CardT}_1 \times \mbox{CardT}_2 \rightarrow \mathbb{B}$\\
\noindent output: $out := \mbox{CardT}_1.\m{getSuit}()=\mbox{CardT}_2.\m{getSuit}()$\\

\noindent BoardEmpty: $\mbox{seq of StackT} \rightarrow \mathbb{B}$\\
\noindent output: $out := \forall(s : \mbox{StackT} | s \in (\mbox{seq of StackT} : \m{StackEmpty}(s))$\\

\noindent FreeCellEmpty: $\mbox{seq of CardT} \rightarrow \mathbb{B}$\\
\noindent output: $out := \forall(c : \mbox{CardT} | c \in \mbox{seq of CardT} : \lnot c.\m{isValid}())$\\

\noindent FoundationComplete: $\mbox{seq of CardT} \rightarrow \mathbb{B}$\\
\noindent output: $out := \forall(c : \mbox{CardT} | c \in \mbox{seq of CardT} : c.\m{getRank}()=\m{KING}) \land \forall(s_1 : \mbox{CardT} | s_1 \in \mbox{seq of CardT} : s_1.\m{getSuit} \neq \mbox{NAS} \land \forall(s_2 : \mbox{CardT} | s_2 \in \mbox{seq of CardT} \mbox{\textbackslash} s_1 : s_1.\m{getSuit}() \neq s_2.\m{getSuit}()))$\\

%\noindent <name>: <mathematical input>\\
%\noindent transition: <spec>

\end{document}
