\documentclass[12pt, titlepage]{article}

\usepackage{enumitem}
\setitemize{noitemsep,topsep=0pt,parsep=0pt,partopsep=0pt}

\usepackage{booktabs}
\usepackage{tabularx}
\usepackage{hyperref}
\hypersetup {
    colorlinks,
    citecolor=black,
    filecolor=black,
    linkcolor=red,
    urlcolor=blue
}
\usepackage[round]{natbib}

\title{SE 3XA3: Test Report\\Title of Project}

\author{Team 20, Staroids
		\\ Eoin Lynagh, lynaghe
		\\ Jason Nagy, nagyj2
		\\ Moziah San Vicente, sanvicem
}

\date{\today}

% \input{../Comments} %

\begin{document}

\maketitle

\pagenumbering{roman}
\tableofcontents
\listoftables
\listoffigures

\begin{table}[bp]
\caption{\bf Revision History}
\begin{tabularx}{\textwidth}{p{3cm}p{2cm}X}
\toprule {\bf Date} & {\bf Version} & {\bf Notes}\\
\midrule
Nov 23 & 0.1 & Added personal info to document\\
Nov 23 & 0.15 & Code coverage, Automated testing, Changes due to testing\\
Nov 23 & 0.2 & Added usability requirement results\\
Nov 23 & 0.25 & \\
Nov 23 & 0.3 & \\
Nov 23 & 0.35 & \\
\bottomrule
\end{tabularx}
\end{table}

\newpage

\pagenumbering{arabic}

This document is meant to show \\

\section{Functional Requirements Evaluation}
\subsection{Tests}
\begin{itemize}
  \item{Loading into pre-game\\}
  \item{Starting game\\}
  \item{Prohibit game start on pause\\}
  \item{Permitting pause\\}
  \item{Displaying player information above all\\}
  \item{Player firing and alien destruction\\}
  \item{Asteroid collision with player\\}
  \item{Asteroid separation\\}
  \item{Witholding player spawn\\}
  \item{Post-game\\}
\end{itemize}
\subsection{Evaluation}
\subsubsection{Loading into pre-game}
...\\
\subsubsection{Starting game}
...\\
\subsubsection{Prohibit game start on pause}
...\\
\subsubsection{Permitting pause}
...\\
\subsubsection{Displaying player information above all}
...\\
\subsubsection{Player firing and alien destruction}
...\\
\subsubsection{Asteroid collision with player}
...\\
\subsubsection{Asteroid separation}
Every time the player character is hit by an enemy, the lives count shall decrease by one.\\
...\\
\subsubsection{Witholding player spawn}
...\\
\subsubsection{Post-game}
...\\

%=-=-=-=-=-=-=-=-=-=-=-=-=-=-=-=-=-=-=-=-=-=-=-=-=-=-=-=-=-=-=-=-=-=-=-=-=-=-=-%

%=-=-=-=-=-=-=-=-=-=-=-=-=-=-=-=-=-=-=-=-=-=-=-=-=-=-=-=-=-=-=-=-=-=-=-=-=-=-=-%

\section{Nonfunctional Requirements Evaluation}

\subsection{Usability}

\subsubsection{Player test}
To test usability, playability and feel of the game, the Staroids team recruited ten outside people to play Staroids and give feedback. Several questions were asked of the test players, particularly about the feel of the game and how it played. The most common feedback we recieved related to the alien and the appearance of the ship. The alien was annoying and not a challenge to the play testers, so his spawn chance and times were reduced and he was made easier to hit. When the testers faced the new alien, there were much fewer complaints about him. The second large complaint from testers was the appearance of the ship. The ship, alien and asteroids used to be all white filled shapes with a black border. Testers often could not tell the direction the ship was facing, or even which object was their ship. To fix this, the ship was made narrower which helped show the front of the ship and the player ship and alien were made into solid black shapes so they are visible at a glance. The alien bullets and player bullets were more differentiated due to feedback as well.\\

\subsubsection{Browser test}
To test Staroids, the Staroids team used all major browsers for development and testing in an effort to notice Staroids bugs and issues that were specific to one browser. None were found.

\subsection{Informational}
%Player information

\subsection{Internal}
%JSDoc Documentation Inspection
%JSDoc Documentation Compilation
%Complicated Function Documentation Inspection


\section{Comparison to Existing Implementation}

This section will not be appropriate for every project.

\section{Unit Testing}

\section{Changes Due to Testing}
So far there have been a couple of major changes due to testing: the alien movement, player movement, game sounds and player colour. The alien changes came about when during testing  alien was moving in the pattern desired (sine wave) but the amplitude was too small, and did not give the player enough time to get out of the way, so it was adjusted to make it's speed lower as well as its amplitude higher so it covered a larger path. The next change that followed was desirable pattern accros the screen in that of a sine wave,

\section{Automated Testing}
In

\section{Trace to Requirements}

\section{Trace to Modules}

\section{Code Coverage Metrics}
No formal code coverage software was used for this project based on the decision made by the team to move away from automated testing as explained above and rely strictly on manual. However throughout the development of this project specifically during the MIS portion the team went through all individual functions, variables, objects and methods for this program and either documented them and what they do/how they are used or deleted them if they were not used. This has helped increase the percentage of code covered in during testing, as well as an easy way to just run throguh the description of the documented objects and functions to manually see if they are covered in our tests.


\bibliographystyle{plainnat}

\bibliography{SRS}

\end{document}
