\documentclass{article}

\usepackage{tabularx}
\usepackage{booktabs}

\title{SE 3XA3: Problem Statement\\Staroids}

\author{Team 20, Staroids
		\\ Moziah San Vicente, 400091284, sanvicem
		\\ Eoin Lynagh, 400067675, lynaghe
		\\ Jason Nagy, 400055130, nagyj2
}

\date{Friday September 21, 2018}

%\input{../Comments}%

\begin{document}

\begin{table}[hp]
\caption{Revision History} \label{TblRevisionHistory}
\begin{tabularx}{\textwidth}{llX}
\toprule
\textbf{Date} & \textbf{Developer(s)} & \textbf{Change}\\
\midrule
Sept 20 & Jason Nagy & Added basics\\
Date2 & Name(s) & Description of changes\\
... & ... & ...\\
\bottomrule
\end{tabularx}
\end{table}

\newpage

\maketitle

Vision: Planning, designing and implementing a large piece of software is an expensive and
daunting task. However, for games there is less of a focus on the documentation because the
software is not meant to be altered by the user, but that doesn't mean that other programmers
will not try to edit and revise the game to make it better or their own versions.
This is true in Doug McInnes' HTML5-Asteroids program, a version of the arcade game Asteroids
was developed for a user to enjoy. However, when looking at the code, it is a convoluted mess of
functions and statements making it very difficult for a programmer to understand and work on
if they desired to make edits. If the game were designed more in accordance to software desgin
principles, the game would be better to both the user and other developers.
\\Issue Statement: While functional, HTML5-Asteroids lacks the process structure and
documentation of a large software project. This makes edits, revisons, and additions to the
code more difficult as well as making the code less readable for other programmers. The game
also does not follow the software engineering principles of modularity or information hiding.
\\Method: By using the proper software design principles and process, the Asteroids game can
be recreated with the future in mind. It will allow for the game to be expanded upon and more
easily comprehended by any programmer that wishes to work on it or understand how the game was
developed.

Needed:
Who are stakeholders? Anyone who enjoys playing older classic games, programmers who want to further expand on our implementation just as we are going to on the original, as well as any online gaming websites that may want to use our implementation are the stakeholders that we're thinking about when developing this re-implementation of the game. \\
Why is this important? Our project is important because the world of video games is a billion dollar industry that pretty much effects everybody, whether it is directly through them playing or through their spouse, partner, child, sibling or cousin. And now with the evolution of smart-phones and mobile gaming it is becoming even bigger. So our team as people who grew up apart of this gaming community wanted to add something to it other than just our dollars in buying games and our time being thrown into playing them. We wanted to re-develop a game to bring people hours of joy just as video games have brought us throughout our lives, and truly give back to the community that has raised us.\\
Environment for software? The environment we our using to create our implementation is javascript, which can be run on any web browser, and has ease of access because there is no downloads needed to play, just an internet connection.

%\wss{comment}

%\ds{comment}

%\mj{comment}

%\cm{comment}

%\mh{comment}

\end{document}
