\documentclass{article}

\usepackage{tabularx}
\usepackage{booktabs}

\title{SE 3XA3: Problem Statement\\Staroids}

\author{Team 20, Staroids
		\\ Moziah San Vicente, 400091284, sanvicem
		\\ Eoin Lynagh, 400067675, lynaghe
		\\ Jason Nagy, 400055130, nagyj2
}

\date{Friday September 21, 2018}

%\input{../Comments}%

\begin{document}

\begin{table}[hp]
\caption{Revision History} \label{TblRevisionHistory}
\begin{tabularx}{\textwidth}{llX}
\toprule
\textbf{Date} & \textbf{Developer(s)} & \textbf{Change}\\
\midrule
Sept 20 & Jason Nagy & Added basics\\
Sept 20 & Moziah San Vicente & Added stakeholders, importance of project and software environment\\
Sept 21 & Jason Nagy & Edited grammar\\
\bottomrule
\end{tabularx}
\end{table}

\newpage

\maketitle

Planning, designing and implementing a large piece of software is an expensive and
daunting task. However, for games there is less of a focus on the documentation because the
software is not meant to be altered by the user, but that doesn't mean that other programmers
will not try to edit and revise the game to make it better or their own versions.
This is true in Doug McInnes' HTML5-Asteroids program, a version of the arcade game Asteroids
was developed for a user to enjoy. However, when looking at the code, it is a convoluted mess of
functions and statements making it very difficult for a programmer to understand and work on
if they desired to make edits. If the game were designed more in accordance to software desgin
principles, the game would be better to both the user and other developers.\\
While functional, HTML5-Asteroids lacks the process structure and
documentation of a large software project. This makes edits, revisons, and additions to the
code more difficult as well as making the code less readable for other programmers. The game
also does not follow the software engineering principles of modularity or information hiding.\\
By using the proper software design principles and process, the Asteroids game can
be recreated with the future in mind. It will allow for the game to be expanded upon and more
easily comprehended by any programmer that wishes to work on it or understand how the game was
developed.\\
The stakeholders in this project include anyone who enjoys playing older classic games, programmers
who want to further expand on both systems-as-is and the system-to-be, as well as any online gaming
websites that may want to use our implementation.
The video game industry is a billion dollar business that effects everyone, whether it is directly
through playing games or through their spouse, partner, child, sibling or cousin. Now with the
evolution of smart-phones and mobile gaming, the industry is becoming even bigger.\\
Furthermore, the environment that the system-to-be will embody is JavaScript, which can be run on
any web browser. JavaScript is also convinient because there is no downloads needed to play,
just an internet connection.

%\wss{comment}

%\ds{comment}

%\mj{comment}

%\cm{comment}

%\mh{comment}

\end{document}
